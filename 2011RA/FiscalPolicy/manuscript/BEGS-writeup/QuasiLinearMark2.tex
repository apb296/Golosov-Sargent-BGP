%2multibyte Version: 5.50.0.2953 CodePage: 1251
%%              Scientific Word   Wrap/Unwrap  Version 2.5             %
%%              Scientific Word   Wrap/Unwrap  Version 3.0             %
%% If you are separating the files in this message by hand, you will   %
%% need to identify the file type and place it in the appropriate      %
%% directory.  The possible types are: Document, DocAssoc, Other,      %
%% Macro, Style, Graphic, PastedPict, and PlotPict. Extract files      %
%% tagged as Document, DocAssoc, or Other into your TeX source file    %
%% directory.  Macro files go into your TeX macros directory. Style    %
%% files are used by Scientific Word and do not need to be extracted.  %
%% Graphic, PastedPict, and PlotPict files should be placed in a       %
%% graphics directory.                                                 %
%% Graphic files need to be converted from the text format (this is    %
%% done for e-mail compatability) to the original 8-bit binary format. %
%% Files included:                                                     %
%% "/document/NonbalancedJune18.tex", Document, 124302, 6/18/2004, 0:45:10, ""%
%% "/document/HZH9JA00.wmf", PastePict, 30602, 6/17/2004, 21:53:40, "" %
%% "/document/HZH9JA01.wmf", PastePict, 31310, 6/17/2004, 21:55:05, "" %
%% "/document/HZH9JA02.wmf", PastePict, 31498, 6/17/2004, 21:56:09, "" %
%% "/document/HXIWF402.wmf", ImportPict, 31112, 6/11/2004, 2:52:58, "" %
%% "/document/figure1new.wmf", ImportPict, 8034, 6/11/2004, 2:52:58, ""%
%% "/document/HZH9JA03.wmf", PastePict, 49212, 6/11/2004, 2:52:58, ""  %
%%%%%%%%%%%%%%%%% Start /document/NonbalancedJune18.tex %%%%%%%%%%%%%%%%
%%TCIDATA{OutputFilter=LATEX.DLL}
%%TCIDATA{Version=5.50.0.2890}
%%TCIDATA{Codepage=1251}
%%TCIDATA{<META NAME="SaveForMode" CONTENT="1">}
%%TCIDATA{BibliographyScheme=Manual}
%%TCIDATA{Created=Sat Jul 31 12:49:26 1999}
%%TCIDATA{LastRevised=Monday, February 20, 2012 12:05:46}
%%TCIDATA{<META NAME="GraphicsSave" CONTENT="32">}
%%TCIDATA{Language=American English}
%%TCIDATA{CSTFile=LaTeX article (bright).cst}
%=====================================================  tom front end %==================================================
%\documentclass[12pt]{article}
%\usepackage{amsmath,amssymb,amsthm,enumerate,graphicx}
%\usepackage{ifthen,latexsym,syntonly}
%\usepackage{setspace}
%\usepackage[showrefs]{refcheck}  %use this to show equation and section labels
%\usepackage{color}
%\usepackage[round,comma,authoryear]{natbib}   % for natbib
%\usepackage{subfigure}
%\usepackage{float}
%\bibliographystyle{mynat}
%\onehalfspacing
%   % for natbib
%%\bibpunct{(}{)}{,}{a}{}{,}  % for natbib
%%                            % need to have mynat.bst in an accesssible directory
%\bibpunct{(}{)}{,}{a}{}{,}  % for natbib
%                            % need to have mynat.bst in an accesssible directory
%\newtheorem{theorem}{Theorem}[section]
%\newtheorem{remark}[theorem]{Remark}
%\newtheorem{assumption}[theorem]{Assumption}
%\newtheorem{case}[theorem]{Case}
%\newtheorem{claim}[theorem]{Claim}
%%\newtheorem{conclusion}[theorem]{Conclusion}
%\newtheorem{corollary}[theorem]{Corollary}
%\newtheorem{condition}[theorem]{Condition}
%\newtheorem{criterion}[theorem]{Criterion}
%\newtheorem{definition}[theorem]{Definition}
%\newtheorem{example}[theorem]{Example}
%\newtheorem{lemma}[theorem]{Lemma}
%\newtheorem{problem}[theorem]{Problem}
%\newtheorem{proposition}[theorem]{Proposition}
%%\newtheorem{solution}[theorem]{Solution}
%%\newtheorem{summary}[theorem]{Summary}
%\newtheorem{thm}[theorem]{Theorem}
%\setlength{\oddsidemargin}{.05in} \setlength{\topmargin}{-.45in}
%\setlength{\textwidth}{6.4in} \setlength{\textheight}{8.5in}
%%\pagestyle{empty}
%%=================================================== end of tom front end %=================================================
%\usepackage[showrefs]{refcheck}
%\usepackage{showlabels}
%\input{tcilatex}


\documentclass[thmsb,11pt]{article}
%%%%%%%%%%%%%%%%%%%%%%%%%%%%%%%%%%%%%%%%%%%%%%%%%%%%%%%%%%%%%%%%%%%%%%%%%%%%%%%%%%%%%%%%%%%%%%%%%%%%%%%%%%%%%%%%%%%%%%%%%%%%%%%%%%%%%%%%%%%%%%%%%%%%%%%%%%%%%%%%%%%%%%%%%%%%%%%%%%%%%%%%%%%%%%%%%%%%%%%%%%%%%%%%%%%%%%%%%%%%%%%%%%%%%%%%%%%%%%%%%%%%%%%%%%%%
\usepackage{amsfonts}
\usepackage{amssymb}
\usepackage{amsmath}
\usepackage{graphicx}
\usepackage{color}
\usepackage{refcount}
\usepackage{natbib}
\usepackage{hyperref}
\setcounter{MaxMatrixCols}{10}
%TCIDATA{TCIstyle=article/art4.lat,lart,article}

%TCIDATA{OutputFilter=LATEX.DLL}
%TCIDATA{Version=5.50.0.2953}
%TCIDATA{Codepage=1251}
%TCIDATA{<META NAME="SaveForMode" CONTENT="1">}
%TCIDATA{BibliographyScheme=Manual}
%TCIDATA{Created=Sat Jul 31 12:49:26 1999}
%TCIDATA{LastRevised=Wednesday, June 12, 2013 12:37:50}
%TCIDATA{<META NAME="GraphicsSave" CONTENT="32">}
%TCIDATA{Language=American English}
%TCIDATA{CSTFile=LaTeX article (bright).cst}

\newtheorem{theorem}{Theorem}
\newtheorem{acknowledgement}[theorem]{Acknowledgement}
\newtheorem{algorithm}[theorem]{Algorithm}
\newtheorem{assumption}{Assumption}
\newtheorem{axiom}{Axiom}
\newtheorem{case}[theorem]{Case}
\newtheorem{claim}[theorem]{Claim}
\newtheorem{conclusion}[theorem]{Conclusion}
\newtheorem{condition}[theorem]{Condition}
\newtheorem{conjecture}{Conjecture}
\newtheorem{corollary}{Corollary}
\newtheorem{criterion}[theorem]{Criterion}
\newtheorem{definition}{Definition}
\newtheorem{lemma}{Lemma}
\newtheorem{problem}[theorem]{Problem}
\newtheorem{proposition}{Proposition}
\newtheorem{solution}[theorem]{Solution}
\newtheorem{summary}[theorem]{Summary}
\newtheorem{example}{Example}
\newtheorem{exercise}{Exercise}
\newtheorem{notation}{Notation}
\newtheorem{remark}{Remark}
\newcommand{\bmat}{\begin{matrix}}
\newcommand{\emat}{\end{matrix}}
\newcommand{\ov}{\overline}
\newcommand{\un}{\underline}
\newenvironment{proof}[1][Proof]{\noindent \textbf{#1.} }{\  \rule{0.5em}{0.5em}}
\topmargin=-1cm
\oddsidemargin=-0cm
\textheight=22.2cm
\textwidth=16cm
\setcounter{secnumdepth}{2}
\pagestyle{plain}
\setcounter{figure}{0}
\sloppy
\input{sw20elba.sty}
\input{tcilatex}
\begin{document}


\smallskip Let $\alpha _{2}$ be the Pareto weight on the unproductive agent.
Given a (initial) net distribution of assets $\tilde{b}_{2}=b_{2}-b_{1}$ the
optimal solution solves the following Bellman equation.

\begin{equation}
V(\beta^{-1}\tilde{b}_2,s\_)=\max_{c_1(s),c_2(s),l_1(s),\tilde{b}_2^{\prime }(s),\chi(s)}
\sum_{s}\Pr(s|s\_) \left \{ (1-\alpha_2)\left(c_1(s)-h[l_1(s)]\right)+\alpha_2
c_2(s) +\beta V(\beta^{-1} \tilde{b}^{\prime }_2(s),s)\right\}
\label{eq:QuasiLinearProblem}
\end{equation}

subject to
\begin{subequations}
\begin{equation}
c_2(s)-c_1(s)+\tilde{b}^{\prime }_2(s)+ \frac{l_1(s)h^{\prime }[l_1(s)]}{1+\chi(s)}=\frac{%
\tilde{b}_2}{\beta}  \label{eq:QLImp}
\end{equation}
\begin{equation}
c_2(s)+c_1(s)+g(s)\leq\theta_1l_1(s)  \label{eq:QLResource}
\end{equation}


\begin{equation}
c_1(s)\chi(s)=0\label{eq:QLcompslackness}
\end{equation}

\begin{equation}
c_1(s)\geq 0\quad c_2(s)\geq0  \quad \chi(s)\geq0\label{eq:QLconsbounds}
\end{equation}

\begin{equation}
\tilde{b}_2^{\prime }(s)\in [\underline b,\bar b]
\end{equation}

Let $\mu(s)\Pr(s|s\_) , \xi(s)\Pr(s|s\_)$ , $\lambda^{c_i}(s)\Pr(s|s\_)$ be the
multipliers on the implementability (\ref{eq:QLImp}), resource constraint (\ref{eq:QLResource}) and the non-negativity constraints (\ref{eq:QLconsbounds})
for the respective consumption. Let $\lambda^{cs}(s)Pr(s|s\_),\lambda^{\chi}(s)Pr(s|s\_)$ be the multipliers on the completmentary slackness \ref{eq:QLcompslackness} and non-negativity of the Agent's multiplier $\chi(s)$.

For now we assume that $\underline b$ and $%
\bar b$ are natural debt limits and ignore them on the equilibrium path. The
FONCs of the problem are
\end{subequations}

\begin{subequations}
\begin{equation}
1-\alpha_2+\mu(s)+ \lambda^{c_1}(s)-\xi(s)+\lambda^{cs}(s)\chi(s)=0
\end{equation}
\begin{equation}
\alpha_2 -\mu(s)+ \lambda^{c_2}(s)-\xi(s)=0
\end{equation}

Let $g(l)=lh^{\prime \prime }(l)+h^{\prime }(l)$
\begin{equation}
-(1-\alpha_2)h^{\prime }[l_1(s)]-\mu(s)\frac{g[l_1(s)]}{1+\chi(s)}+\theta_1\xi(s)=0
\end{equation}

\begin{equation}
\mu(s)\frac{l_1(s)h'[l_1(s)]}{(1+\chi(s))^2}+\lambda^{cs}(s)c)_1(s)+\lambda^{\chi}(s)=0 \label{eq:focchi}
\end{equation}


\begin{equation}
\mu(s)=V_{\tilde{b}_2}[\tilde{b}^{\prime }_2(s),s]
\end{equation}
and lastly the Envelope theorem gives us

\begin{equation}
\mathbb{E}_{s\_} V_{\tilde{b}_2}[\tilde{b}^{\prime }_2(s),s]=V_{\tilde{b}_2}[%
\tilde{b}_2,s\_]
\end{equation}

For an interior solution,i.e $c_1(s)>0, c_2(s)>0$, we have constant labor
supply that solves the following equation

\end{subequations}

\begin{equation}
-(1-\alpha_2)h^{\prime }(l_1)-(\alpha_2-\frac{1}{2})g(l_1)+\frac{\theta_1}{2}%
=0
\end{equation}

For CES labor disutility, $h(l)=\frac{l^{1+\gamma}}{1+\gamma}$, this yields

\begin{equation}
l^{*}_1\left(\alpha_2\right)=\left(\frac{\theta_1}{2\left[\left(\alpha_2-%
\frac{1}{2}\right)(1+\gamma)+(1-\alpha_2) \right ]}\right)^{\frac{1}{\gamma}}
\label{eq:QLLabor}
\end{equation}

The condition $\alpha_2\geq \frac{(\gamma-1)}{2\gamma}$ ensures that $%
l_1^*\left(\alpha_2\right)\geq0$

We can use the implementability constraint (\ref{eq:QLImp}) with the guess $%
\tilde{b}^{\prime }_2(s)=\tilde{b}_2$ and the resource constraint (\ref%
{eq:QLResource}) to back out consumptions for each agent.

\begin{subequations}
\begin{equation}
c_1(s)=\frac{1}{2}\left[ \theta_1l_1^*\left(\alpha_2\right)
+l_1^*\left(\alpha_2\right)h^{\prime }(l_1^*\left(\alpha_2\right))-\tilde{b}%
_2 \left(\frac{1}{\beta}-1\right)-g(s)\right]
\end{equation}
\begin{equation}
c_2(s)=\frac{1}{2}\left[ \theta_1l_1^*\left(\alpha_2\right)
-l_1^*\left(\alpha_2\right)h^{\prime }(l_1^*\left(\alpha_2\right))+\tilde{b}%
_2 \left(\frac{1}{\beta}-1\right)-g(s)\right]
\end{equation}

To be a valid interior solution we need $\tilde{b}_2\in \left[ \underline{%
\mathcal{B}}_2\left(\alpha_2\right) , \bar{\mathcal{B}}_2\left(\alpha_2%
\right) \right]$ where this interval is given by

\end{subequations}

\begin{subequations}
\begin{equation}
\underline{\mathcal{B}}_2\left(\alpha_2\right)=\frac{\max_{s}g(s)-\theta_1
l_1^*\left(\alpha_2\right)+l_1^*\left(\alpha_2\right)h^{\prime
}(l_1^*\left(\alpha_2\right))}{\frac{1}{\beta}-1}
\end{equation}
\begin{equation}
\label{eq:QLLowerLimit}
\bar{\mathcal{B}}_2\left(\alpha_2\right)=\frac{-\max_{s}g(s)+\theta_1
l_1^*\left(\alpha_2\right)+l_1^*\left(\alpha_2\right)h^{\prime
}(l_1^*\left(\alpha_2\right))}{\frac{1}{\beta}-1}
\end{equation}
\end{subequations}

The condition for non-empty interior is a very natural one $%
-\max_{s}g(s)+\theta_1 l_1^*\left(\alpha_2\right)>0$, that says that output
sufficient to cover the worst possible realization of government consumption.



For $\tilde{b}_2>\underline{\mathcal{B}_2}$, let us consider an alternative solution to the FOCs where $\chi(s)>0,\lambda^{c2}(s)>0$. Note that $\chi(s)>0$ implies $c_1(s)=0$. The foc with $\chi(s)$, \ref{eq:focchi} tells us that $\mu(s)=0$. We have the following allocation

\begin{align}
l_1(s)=l^{FB}=\left(\frac{\theta_1\alpha_2}{1-\alpha_2}\right)^{\frac{1}{\gamma}} 
\end{align}
\begin{align}
c_2(s)=\theta_1l^{FB}-g(s)
\end{align}

\begin{align}
\tilde{b}'_2(s)=\frac{\tilde{b}_2}{\beta}-\frac{(l^{FB})^{1+\gamma}}{1+\chi(s)}-\theta_1l^{FB}+g(s)
\end{align}

Note that $\lambda^{c1}(s)+\lambda^{cs}(s)\chi(s)$ is a martingale. \footnote{Since I can sign $\lambda^{cs}(s)$, I dont know if it is bounded from below. Otherwise we could argue that it converges to a constant and then figure out if that constant is zero or not}



WANT operator :

\begin{itemize}
 \item The best thing would be to show that $\lim_t \chi(s^t)=0$. This would mean that eventually we can ignore the impact of these constraints.
 \item There exists a $\hat{\mathcal{B}}$ such that $\chi(s)=0$ for all $\tilde{b}_{2}\leq\hat{\mathcal{B}}$ and $\underline{\mathcal{B}}(\alpha_2)<\hat{\mathcal{B}}$. If this is true then we can apply the old proof for convergence on one side i.e if $\tilde{b}_{2,-1}<\underline{\mathcal{B}}(\alpha_2)$, then $\lim_t\tilde{b}_{2,t}=\underline{\mathcal{B}}(\alpha_2)$
\end{itemize}




\end{document}
